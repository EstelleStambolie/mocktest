\documentclass[11pt,preprint, authoryear]{elsarticle}

\usepackage{lmodern}
%%%% My spacing
\usepackage{setspace}
\setstretch{1.2}
\DeclareMathSizes{12}{14}{10}{10}

% Wrap around which gives all figures included the [H] command, or places it "here". This can be tedious to code in Rmarkdown.
\usepackage{float}
\let\origfigure\figure
\let\endorigfigure\endfigure
\renewenvironment{figure}[1][2] {
    \expandafter\origfigure\expandafter[H]
} {
    \endorigfigure
}

\let\origtable\table
\let\endorigtable\endtable
\renewenvironment{table}[1][2] {
    \expandafter\origtable\expandafter[H]
} {
    \endorigtable
}


\usepackage{ifxetex,ifluatex}
\usepackage{fixltx2e} % provides \textsubscript
\ifnum 0\ifxetex 1\fi\ifluatex 1\fi=0 % if pdftex
  \usepackage[T1]{fontenc}
  \usepackage[utf8]{inputenc}
\else % if luatex or xelatex
  \ifxetex
    \usepackage{mathspec}
    \usepackage{xltxtra,xunicode}
  \else
    \usepackage{fontspec}
  \fi
  \defaultfontfeatures{Mapping=tex-text,Scale=MatchLowercase}
  \newcommand{\euro}{€}
\fi

\usepackage{amssymb, amsmath, amsthm, amsfonts}

\def\bibsection{\section*{References}} %%% Make "References" appear before bibliography


\usepackage[round]{natbib}

\usepackage{longtable}
\usepackage[margin=2.3cm,bottom=2cm,top=2.5cm, includefoot]{geometry}
\usepackage{fancyhdr}
\usepackage[bottom, hang, flushmargin]{footmisc}
\usepackage{graphicx}
\numberwithin{equation}{section}
\numberwithin{figure}{section}
\numberwithin{table}{section}
\setlength{\parindent}{0cm}
\setlength{\parskip}{1.3ex plus 0.5ex minus 0.3ex}
\usepackage{textcomp}
\renewcommand{\headrulewidth}{0.2pt}
\renewcommand{\footrulewidth}{0.3pt}

\usepackage{array}
\newcolumntype{x}[1]{>{\centering\arraybackslash\hspace{0pt}}p{#1}}

%%%%  Remove the "preprint submitted to" part. Don't worry about this either, it just looks better without it:
\makeatletter
\def\ps@pprintTitle{%
  \let\@oddhead\@empty
  \let\@evenhead\@empty
  \let\@oddfoot\@empty
  \let\@evenfoot\@oddfoot
}
\makeatother

 \def\tightlist{} % This allows for subbullets!

\usepackage{hyperref}
\hypersetup{breaklinks=true,
            bookmarks=true,
            colorlinks=true,
            citecolor=blue,
            urlcolor=blue,
            linkcolor=blue,
            pdfborder={0 0 0}}


% The following packages allow huxtable to work:
\usepackage{siunitx}
\usepackage{multirow}
\usepackage{hhline}
\usepackage{calc}
\usepackage{tabularx}
\usepackage{booktabs}
\usepackage{caption}


\newenvironment{columns}[1][]{}{}

\newenvironment{column}[1]{\begin{minipage}{#1}\ignorespaces}{%
\end{minipage}
\ifhmode\unskip\fi
\aftergroup\useignorespacesandallpars}

\def\useignorespacesandallpars#1\ignorespaces\fi{%
#1\fi\ignorespacesandallpars}

\makeatletter
\def\ignorespacesandallpars{%
  \@ifnextchar\par
    {\expandafter\ignorespacesandallpars\@gobble}%
    {}%
}
\makeatother

\newlength{\cslhangindent}
\setlength{\cslhangindent}{1.5em}
\newenvironment{CSLReferences}%
  {\setlength{\parindent}{0pt}%
  \everypar{\setlength{\hangindent}{\cslhangindent}}\ignorespaces}%
  {\par}


\urlstyle{same}  % don't use monospace font for urls
\setlength{\parindent}{0pt}
\setlength{\parskip}{6pt plus 2pt minus 1pt}
\setlength{\emergencystretch}{3em}  % prevent overfull lines
\setcounter{secnumdepth}{5}

%%% Use protect on footnotes to avoid problems with footnotes in titles
\let\rmarkdownfootnote\footnote%
\def\footnote{\protect\rmarkdownfootnote}
\IfFileExists{upquote.sty}{\usepackage{upquote}}{}

%%% Include extra packages specified by user
\usepackage{array}
\usepackage{caption}
\usepackage{graphicx}
\usepackage{siunitx}
\usepackage[normalem]{ulem}
\usepackage{colortbl}
\usepackage{multirow}
\usepackage{hhline}
\usepackage{calc}
\usepackage{tabularx}
\usepackage{threeparttable}
\usepackage{wrapfig}
\usepackage{adjustbox}
\usepackage{hyperref}

%%% Hard setting column skips for reports - this ensures greater consistency and control over the length settings in the document.
%% page layout
%% paragraphs
\setlength{\baselineskip}{12pt plus 0pt minus 0pt}
\setlength{\parskip}{12pt plus 0pt minus 0pt}
\setlength{\parindent}{0pt plus 0pt minus 0pt}
%% floats
\setlength{\floatsep}{12pt plus 0 pt minus 0pt}
\setlength{\textfloatsep}{20pt plus 0pt minus 0pt}
\setlength{\intextsep}{14pt plus 0pt minus 0pt}
\setlength{\dbltextfloatsep}{20pt plus 0pt minus 0pt}
\setlength{\dblfloatsep}{14pt plus 0pt minus 0pt}
%% maths
\setlength{\abovedisplayskip}{12pt plus 0pt minus 0pt}
\setlength{\belowdisplayskip}{12pt plus 0pt minus 0pt}
%% lists
\setlength{\topsep}{10pt plus 0pt minus 0pt}
\setlength{\partopsep}{3pt plus 0pt minus 0pt}
\setlength{\itemsep}{5pt plus 0pt minus 0pt}
\setlength{\labelsep}{8mm plus 0mm minus 0mm}
\setlength{\parsep}{\the\parskip}
\setlength{\listparindent}{\the\parindent}
%% verbatim
\setlength{\fboxsep}{5pt plus 0pt minus 0pt}



\begin{document}



\begin{frontmatter}  %

\title{What really buys happiness}

% Set to FALSE if wanting to remove title (for submission)




\author[Add1]{Estelle Stambolie\footnote{\textbf{Contributions:}
  \newline \emph{The authors would like to thank no institution for
  money donated to this project. Thank you sincerely.}}}
\ead{}

\author[Add1,Add2]{John Smith}
\ead{John@gmail.com}

\author[Add1,Add2]{John Doe}
\ead{Joe@gmail.com}



\address[Add1]{Student,Stellenbosch, South Africa}
\address[Add2]{Some other Institution, Cape Town, South Africa}

\cortext[cor]{Corresponding author: Estelle Stambolie\footnote{\textbf{Contributions:}
  \newline \emph{The authors would like to thank no institution for
  money donated to this project. Thank you sincerely.}}}

\begin{abstract}
\small{
Abstract to be written here. The abstract should not be too long and
should provide the reader with a good understanding what you are writing
about. Academic papers are not like novels where you keep the reader in
suspense. To be effective in getting others to read your paper, be as
open and concise about your findings here as possible. Ideally, upon
reading your abstract, the reader should feel he / she must read your
paper in entirety.
}
\end{abstract}

\vspace{1cm}

\begin{keyword}
\footnotesize{
Multivariate GARCH \sep Kalman Filter \sep Copula \\ \vspace{0.3cm}
\textit{JEL classification} L250 \sep L100
}
\end{keyword}
\vspace{0.5cm}
\end{frontmatter}



%________________________
% Header and Footers
%%%%%%%%%%%%%%%%%%%%%%%%%%%%%%%%%
\pagestyle{fancy}
\chead{}
\rhead{}
\lfoot{}
\rfoot{\footnotesize Page \thepage}
\lhead{}
%\rfoot{\footnotesize Page \thepage } % "e.g. Page 2"
\cfoot{}

%\setlength\headheight{30pt}
%%%%%%%%%%%%%%%%%%%%%%%%%%%%%%%%%
%________________________

\headsep 35pt % So that header does not go over title




\hypertarget{introduction}{%
\section{\texorpdfstring{Introduction
\label{Introduction}}{Introduction }}\label{introduction}}

Everybody wants to be happy. Today I set forth to present a convincing
argument of what can buy happiness according to World Happiness Data.

\hypertarget{data}{%
\section*{Data}\label{data}}
\addcontentsline{toc}{section}{Data}

Notice how I used the curly brackets and dash to remove the numbering of
the data section.

Discussion of data should be thorough with a table of statistics and
ideally a figure.

In your tempalte folder, you will find a Data and a Code folder. In
order to keep your data files neat, store all of them in your Data
folder. Also, I strongly suggest keeping this Rmd file for writing and
executing commands, not writing out long pieces of data-wrangling. In
the example below, I simply create a ggplot template for scatter plot
consistency. I suggest keeping all your data in a data folder.

To make your graphs look extra nice in latex world, you could use Tikz
device. Replace dev - `png' with `tikz' in the chunk below. Notice this
makes the build time longer and produces extra tex files - so if you are
comfortable with this, set your device to Tikz and try it out:

\hypertarget{although-the-functions-above-are-really-simple-the-principle-is-simple-containing-calculations-and-data-wrangling-in-their-own-functions-will-make-this-template-much-cleaner-and-more-manageable.}{%
\section{Although the functions above are really simple, the principle
is simple: containing calculations and data wrangling in their own
functions will make this template much cleaner and more
manageable.}\label{although-the-functions-above-are-really-simple-the-principle-is-simple-containing-calculations-and-data-wrangling-in-their-own-functions-will-make-this-template-much-cleaner-and-more-manageable.}}

\hypertarget{when-you-start-working-delete-these-meaningless-functions-and-replace-with-your-own}{%
\section{When you start working, delete these meaningless functions and
replace with your
own\ldots{}}\label{when-you-start-working-delete-these-meaningless-functions-and-replace-with-your-own}}

\begin{verbatim}
To reference the plot above, add a ``\\label'' after the caption in the chunk heading, as done above. Then reference the plot as such: As can be seen, Figures \ref{Figure1}  and \ref{Figure2} are excellent, with Figure \ref{Figure2} being particularly aesthetically pleasing due to its device setting of Tikz. The nice thing now is that it correctly numbers all your figures (and sections or tables) and will update if it moves. The links are also dynamic.


<!-- :::::: {.columns data-latex="[T]"} -->
<!-- ::: {.column data-latex="{0.7\textwidth}"} -->
<!-- ```{r, echo=FALSE, fig.width=4, fig.height=4} -->
<!-- par(mar = c(4, 4, .2, .1)) -->
<!-- plot(cars, pch = 19) -->
<!-- ``` -->
<!-- ::: -->
<!-- ::: {.column data-latex="{0.05\textwidth}"} -->
<!-- \ -->
<!-- ::: -->
<!-- ::: {.column data-latex="{0.2\textwidth}"} -->
<!-- \scriptsize -->

<!-- ## Data {-} -->
<!-- The figure on the left-hand side shows the `cars` data. -->

<!-- Lorem ipsum dolor sit amet, consectetur adipiscing elit, sed do -->
<!-- eiusmod tempor incididunt ut labore et dolore magna aliqua. Ut -->
<!-- enim ad minim veniam, quis nostrud exercitation ullamco laboris -->
<!-- nisi ut aliquip ex ea commodo consequat. -->
<!-- ::: -->
<!-- :::::: -->


## Subsection
Ideally do not overuse subsections. It equates to bad writing.^[This is an example of a footnote by the way. Something that should also not be overused.]


<!-- $$ -->
<!-- This is a commented out section in the writing part. -->
<!-- Comments are created by highlighting text, amnd pressing CTL+C -->
<!-- \\begin{align} -->
<!-- \\beta = \\alpha^2 -->
<!-- \end{align} -->
<!-- $$ -->

# Results

Tables can be included as follows. Use the _xtable_ (or kable) package for tables. Table placement = H implies Latex tries to place the table Here, and not on a new page (there are, however, very many ways to skin this cat. Luckily there are many forums online!).


\begin{table}[H]
\centering
\begin{tabular}{rrrrrrrrrrrr}
  \hline
 & mpg & cyl & disp & hp & drat & wt & qsec & vs & am & gear & carb \\ 
  \hline
1 & 21.00 & 6.00 & 160.00 & 110.00 & 3.90 & 2.62 & 16.46 & 0.00 & 1.00 & 4.00 & 4.00 \\ 
  2 & 21.00 & 6.00 & 160.00 & 110.00 & 3.90 & 2.88 & 17.02 & 0.00 & 1.00 & 4.00 & 4.00 \\ 
  3 & 22.80 & 4.00 & 108.00 & 93.00 & 3.85 & 2.32 & 18.61 & 1.00 & 1.00 & 4.00 & 1.00 \\ 
  4 & 21.40 & 6.00 & 258.00 & 110.00 & 3.08 & 3.21 & 19.44 & 1.00 & 0.00 & 3.00 & 1.00 \\ 
  5 & 18.70 & 8.00 & 360.00 & 175.00 & 3.15 & 3.44 & 17.02 & 0.00 & 0.00 & 3.00 & 2.00 \\ 
   \hline
\end{tabular}
\caption{Short Table Example \label{tab1}} 
\end{table}

To reference calculations __in text__, _do this:_ From table \ref{tab1} we see the average value of mpg is 20.98.

Including tables that span across pages, use the following (note that I add below the table: ``continue on the next page''). This is a neat way of splitting your table across a page.

Use the following default settings to build your own possibly long tables. Note that the following will fit on one page if it can, but cleanly spreads over multiple pages:

\begingroup\fontsize{12pt}{13pt}\selectfont
\begin{longtable}{rrrrrrrrrrr}
\caption{Long Table Example} \\ 
  \toprule
mpg & cyl & disp & hp & drat & wt & qsec & vs & am & gear & carb \\ 
  \hline 
\endhead 
\hline 
{\footnotesize Continued on next page} 
\endfoot 
\endlastfoot 
 \midrule
21.00 & 6.00 & 160.00 & 110.00 & 3.90 & 2.62 & 16.46 & 0.00 & 1.00 & 4.00 & 4.00 \\ 
  21.00 & 6.00 & 160.00 & 110.00 & 3.90 & 2.88 & 17.02 & 0.00 & 1.00 & 4.00 & 4.00 \\ 
  22.80 & 4.00 & 108.00 & 93.00 & 3.85 & 2.32 & 18.61 & 1.00 & 1.00 & 4.00 & 1.00 \\ 
  21.40 & 6.00 & 258.00 & 110.00 & 3.08 & 3.21 & 19.44 & 1.00 & 0.00 & 3.00 & 1.00 \\ 
  18.70 & 8.00 & 360.00 & 175.00 & 3.15 & 3.44 & 17.02 & 0.00 & 0.00 & 3.00 & 2.00 \\ 
  18.10 & 6.00 & 225.00 & 105.00 & 2.76 & 3.46 & 20.22 & 1.00 & 0.00 & 3.00 & 1.00 \\ 
  14.30 & 8.00 & 360.00 & 245.00 & 3.21 & 3.57 & 15.84 & 0.00 & 0.00 & 3.00 & 4.00 \\ 
  24.40 & 4.00 & 146.70 & 62.00 & 3.69 & 3.19 & 20.00 & 1.00 & 0.00 & 4.00 & 2.00 \\ 
  22.80 & 4.00 & 140.80 & 95.00 & 3.92 & 3.15 & 22.90 & 1.00 & 0.00 & 4.00 & 2.00 \\ 
  19.20 & 6.00 & 167.60 & 123.00 & 3.92 & 3.44 & 18.30 & 1.00 & 0.00 & 4.00 & 4.00 \\ 
  17.80 & 6.00 & 167.60 & 123.00 & 3.92 & 3.44 & 18.90 & 1.00 & 0.00 & 4.00 & 4.00 \\ 
  16.40 & 8.00 & 275.80 & 180.00 & 3.07 & 4.07 & 17.40 & 0.00 & 0.00 & 3.00 & 3.00 \\ 
  17.30 & 8.00 & 275.80 & 180.00 & 3.07 & 3.73 & 17.60 & 0.00 & 0.00 & 3.00 & 3.00 \\ 
  15.20 & 8.00 & 275.80 & 180.00 & 3.07 & 3.78 & 18.00 & 0.00 & 0.00 & 3.00 & 3.00 \\ 
  10.40 & 8.00 & 472.00 & 205.00 & 2.93 & 5.25 & 17.98 & 0.00 & 0.00 & 3.00 & 4.00 \\ 
  10.40 & 8.00 & 460.00 & 215.00 & 3.00 & 5.42 & 17.82 & 0.00 & 0.00 & 3.00 & 4.00 \\ 
  14.70 & 8.00 & 440.00 & 230.00 & 3.23 & 5.34 & 17.42 & 0.00 & 0.00 & 3.00 & 4.00 \\ 
  32.40 & 4.00 & 78.70 & 66.00 & 4.08 & 2.20 & 19.47 & 1.00 & 1.00 & 4.00 & 1.00 \\ 
  30.40 & 4.00 & 75.70 & 52.00 & 4.93 & 1.61 & 18.52 & 1.00 & 1.00 & 4.00 & 2.00 \\ 
  33.90 & 4.00 & 71.10 & 65.00 & 4.22 & 1.83 & 19.90 & 1.00 & 1.00 & 4.00 & 1.00 \\ 
  21.50 & 4.00 & 120.10 & 97.00 & 3.70 & 2.46 & 20.01 & 1.00 & 0.00 & 3.00 & 1.00 \\ 
  15.50 & 8.00 & 318.00 & 150.00 & 2.76 & 3.52 & 16.87 & 0.00 & 0.00 & 3.00 & 2.00 \\ 
  15.20 & 8.00 & 304.00 & 150.00 & 3.15 & 3.44 & 17.30 & 0.00 & 0.00 & 3.00 & 2.00 \\ 
  13.30 & 8.00 & 350.00 & 245.00 & 3.73 & 3.84 & 15.41 & 0.00 & 0.00 & 3.00 & 4.00 \\ 
  19.20 & 8.00 & 400.00 & 175.00 & 3.08 & 3.85 & 17.05 & 0.00 & 0.00 & 3.00 & 2.00 \\ 
  27.30 & 4.00 & 79.00 & 66.00 & 4.08 & 1.94 & 18.90 & 1.00 & 1.00 & 4.00 & 1.00 \\ 
  26.00 & 4.00 & 120.30 & 91.00 & 4.43 & 2.14 & 16.70 & 0.00 & 1.00 & 5.00 & 2.00 \\ 
  30.40 & 4.00 & 95.10 & 113.00 & 3.77 & 1.51 & 16.90 & 1.00 & 1.00 & 5.00 & 2.00 \\ 
  15.80 & 8.00 & 351.00 & 264.00 & 4.22 & 3.17 & 14.50 & 0.00 & 1.00 & 5.00 & 4.00 \\ 
  19.70 & 6.00 & 145.00 & 175.00 & 3.62 & 2.77 & 15.50 & 0.00 & 1.00 & 5.00 & 6.00 \\ 
  15.00 & 8.00 & 301.00 & 335.00 & 3.54 & 3.57 & 14.60 & 0.00 & 1.00 & 5.00 & 8.00 \\ 
  21.40 & 4.00 & 121.00 & 109.00 & 4.11 & 2.78 & 18.60 & 1.00 & 1.00 & 4.00 & 2.00 \\ 
   \bottomrule
\end{longtable}
\endgroup

\hfill

<!-- hfill can be used to create a space, like here between text and table. -->


## Huxtable

Huxtable is a very nice package for making working with tables between Rmarkdown and Tex easier.

This cost some adjustment to the Tex templates to make it work, but it now works nicely.

See documentation for this package [here](https://hughjonesd.github.io/huxtable/huxtable.html). A particularly nice addition of this package is for making the printing of regression results a joy (see [here](https://hughjonesd.github.io/huxtable/huxtable.html#creating-a-regression-table)). Here follows an example:


If you are eager to use huxtable, comment out the Huxtable table in the Rmd template, and uncomment the colortbl package in your Rmd's root.

Note that I do not include this in the ordinary template, as some latex users have complained it breaks when they build their Rmds (especially those using tidytex - I don't have this problem as I have the full Miktex installed on mine). Up to you, but I strongly recommend installing the package manually and using huxtable. To make this work, uncomment the _Adding additional latex packages_ part in yaml at the top of the Rmd file. Then comment out the huxtable example in the template below this line. Reknit, and enjoy.


```{=latex}
 
  \providecommand{\huxb}[2]{\arrayrulecolor[RGB]{#1}\global\arrayrulewidth=#2pt}
  \providecommand{\huxvb}[2]{\color[RGB]{#1}\vrule width #2pt}
  \providecommand{\huxtpad}[1]{\rule{0pt}{#1}}
  \providecommand{\huxbpad}[1]{\rule[-#1]{0pt}{#1}}

\begin{table}[ht]
\begin{centerbox}
\begin{threeparttable}
\captionsetup{justification=centering,singlelinecheck=off}
\caption{Regression Output}
 \label{Reg01}
\setlength{\tabcolsep}{0pt}
\begin{tabular}{l l l l}


\hhline{>{\huxb{0, 0, 0}{0.8}}->{\huxb{0, 0, 0}{0.8}}->{\huxb{0, 0, 0}{0.8}}->{\huxb{0, 0, 0}{0.8}}-}
\arrayrulecolor{black}

\multicolumn{1}{!{\huxvb{0, 0, 0}{0}}c!{\huxvb{0, 0, 0}{0}}}{\huxtpad{6pt + 1em}\centering \hspace{6pt} {\fontsize{12pt}{14.4pt}\selectfont } \hspace{6pt}\huxbpad{6pt}} &
\multicolumn{1}{c!{\huxvb{0, 0, 0}{0}}}{\huxtpad{6pt + 1em}\centering \hspace{6pt} {\fontsize{12pt}{14.4pt}\selectfont Reg1} \hspace{6pt}\huxbpad{6pt}} &
\multicolumn{1}{c!{\huxvb{0, 0, 0}{0}}}{\huxtpad{6pt + 1em}\centering \hspace{6pt} {\fontsize{12pt}{14.4pt}\selectfont Reg2} \hspace{6pt}\huxbpad{6pt}} &
\multicolumn{1}{c!{\huxvb{0, 0, 0}{0}}}{\huxtpad{6pt + 1em}\centering \hspace{6pt} {\fontsize{12pt}{14.4pt}\selectfont Reg3} \hspace{6pt}\huxbpad{6pt}} \tabularnewline[-0.5pt]


\hhline{>{\huxb{255, 255, 255}{0.4}}->{\huxb{0, 0, 0}{0.4}}->{\huxb{0, 0, 0}{0.4}}->{\huxb{0, 0, 0}{0.4}}-}
\arrayrulecolor{black}

\multicolumn{1}{!{\huxvb{0, 0, 0}{0}}l!{\huxvb{0, 0, 0}{0}}}{\huxtpad{6pt + 1em}\raggedright \hspace{6pt} {\fontsize{12pt}{14.4pt}\selectfont (Intercept)} \hspace{6pt}\huxbpad{6pt}} &
\multicolumn{1}{r!{\huxvb{0, 0, 0}{0}}}{\huxtpad{6pt + 1em}\raggedleft \hspace{6pt} {\fontsize{12pt}{14.4pt}\selectfont -2256.361 ***} \hspace{6pt}\huxbpad{6pt}} &
\multicolumn{1}{r!{\huxvb{0, 0, 0}{0}}}{\huxtpad{6pt + 1em}\raggedleft \hspace{6pt} {\fontsize{12pt}{14.4pt}\selectfont 5763.668 ***} \hspace{6pt}\huxbpad{6pt}} &
\multicolumn{1}{r!{\huxvb{0, 0, 0}{0}}}{\huxtpad{6pt + 1em}\raggedleft \hspace{6pt} {\fontsize{12pt}{14.4pt}\selectfont 4045.333 ***} \hspace{6pt}\huxbpad{6pt}} \tabularnewline[-0.5pt]


\hhline{}
\arrayrulecolor{black}

\multicolumn{1}{!{\huxvb{0, 0, 0}{0}}l!{\huxvb{0, 0, 0}{0}}}{\huxtpad{6pt + 1em}\raggedright \hspace{6pt} {\fontsize{12pt}{14.4pt}\selectfont } \hspace{6pt}\huxbpad{6pt}} &
\multicolumn{1}{r!{\huxvb{0, 0, 0}{0}}}{\huxtpad{6pt + 1em}\raggedleft \hspace{6pt} {\fontsize{12pt}{14.4pt}\selectfont (13.055)~~~} \hspace{6pt}\huxbpad{6pt}} &
\multicolumn{1}{r!{\huxvb{0, 0, 0}{0}}}{\huxtpad{6pt + 1em}\raggedleft \hspace{6pt} {\fontsize{12pt}{14.4pt}\selectfont (740.556)~~~} \hspace{6pt}\huxbpad{6pt}} &
\multicolumn{1}{r!{\huxvb{0, 0, 0}{0}}}{\huxtpad{6pt + 1em}\raggedleft \hspace{6pt} {\fontsize{12pt}{14.4pt}\selectfont (286.205)~~~} \hspace{6pt}\huxbpad{6pt}} \tabularnewline[-0.5pt]


\hhline{}
\arrayrulecolor{black}

\multicolumn{1}{!{\huxvb{0, 0, 0}{0}}l!{\huxvb{0, 0, 0}{0}}}{\huxtpad{6pt + 1em}\raggedright \hspace{6pt} {\fontsize{12pt}{14.4pt}\selectfont carat} \hspace{6pt}\huxbpad{6pt}} &
\multicolumn{1}{r!{\huxvb{0, 0, 0}{0}}}{\huxtpad{6pt + 1em}\raggedleft \hspace{6pt} {\fontsize{12pt}{14.4pt}\selectfont 7756.426 ***} \hspace{6pt}\huxbpad{6pt}} &
\multicolumn{1}{r!{\huxvb{0, 0, 0}{0}}}{\huxtpad{6pt + 1em}\raggedleft \hspace{6pt} {\fontsize{12pt}{14.4pt}\selectfont ~~~~~~~~} \hspace{6pt}\huxbpad{6pt}} &
\multicolumn{1}{r!{\huxvb{0, 0, 0}{0}}}{\huxtpad{6pt + 1em}\raggedleft \hspace{6pt} {\fontsize{12pt}{14.4pt}\selectfont 7765.141 ***} \hspace{6pt}\huxbpad{6pt}} \tabularnewline[-0.5pt]


\hhline{}
\arrayrulecolor{black}

\multicolumn{1}{!{\huxvb{0, 0, 0}{0}}l!{\huxvb{0, 0, 0}{0}}}{\huxtpad{6pt + 1em}\raggedright \hspace{6pt} {\fontsize{12pt}{14.4pt}\selectfont } \hspace{6pt}\huxbpad{6pt}} &
\multicolumn{1}{r!{\huxvb{0, 0, 0}{0}}}{\huxtpad{6pt + 1em}\raggedleft \hspace{6pt} {\fontsize{12pt}{14.4pt}\selectfont (14.067)~~~} \hspace{6pt}\huxbpad{6pt}} &
\multicolumn{1}{r!{\huxvb{0, 0, 0}{0}}}{\huxtpad{6pt + 1em}\raggedleft \hspace{6pt} {\fontsize{12pt}{14.4pt}\selectfont ~~~~~~~~} \hspace{6pt}\huxbpad{6pt}} &
\multicolumn{1}{r!{\huxvb{0, 0, 0}{0}}}{\huxtpad{6pt + 1em}\raggedleft \hspace{6pt} {\fontsize{12pt}{14.4pt}\selectfont (14.009)~~~} \hspace{6pt}\huxbpad{6pt}} \tabularnewline[-0.5pt]


\hhline{}
\arrayrulecolor{black}

\multicolumn{1}{!{\huxvb{0, 0, 0}{0}}l!{\huxvb{0, 0, 0}{0}}}{\huxtpad{6pt + 1em}\raggedright \hspace{6pt} {\fontsize{12pt}{14.4pt}\selectfont depth} \hspace{6pt}\huxbpad{6pt}} &
\multicolumn{1}{r!{\huxvb{0, 0, 0}{0}}}{\huxtpad{6pt + 1em}\raggedleft \hspace{6pt} {\fontsize{12pt}{14.4pt}\selectfont ~~~~~~~~} \hspace{6pt}\huxbpad{6pt}} &
\multicolumn{1}{r!{\huxvb{0, 0, 0}{0}}}{\huxtpad{6pt + 1em}\raggedleft \hspace{6pt} {\fontsize{12pt}{14.4pt}\selectfont -29.650 *~~} \hspace{6pt}\huxbpad{6pt}} &
\multicolumn{1}{r!{\huxvb{0, 0, 0}{0}}}{\huxtpad{6pt + 1em}\raggedleft \hspace{6pt} {\fontsize{12pt}{14.4pt}\selectfont -102.165 ***} \hspace{6pt}\huxbpad{6pt}} \tabularnewline[-0.5pt]


\hhline{}
\arrayrulecolor{black}

\multicolumn{1}{!{\huxvb{0, 0, 0}{0}}l!{\huxvb{0, 0, 0}{0}}}{\huxtpad{6pt + 1em}\raggedright \hspace{6pt} {\fontsize{12pt}{14.4pt}\selectfont } \hspace{6pt}\huxbpad{6pt}} &
\multicolumn{1}{r!{\huxvb{0, 0, 0}{0}}}{\huxtpad{6pt + 1em}\raggedleft \hspace{6pt} {\fontsize{12pt}{14.4pt}\selectfont ~~~~~~~~} \hspace{6pt}\huxbpad{6pt}} &
\multicolumn{1}{r!{\huxvb{0, 0, 0}{0}}}{\huxtpad{6pt + 1em}\raggedleft \hspace{6pt} {\fontsize{12pt}{14.4pt}\selectfont (11.990)~~~} \hspace{6pt}\huxbpad{6pt}} &
\multicolumn{1}{r!{\huxvb{0, 0, 0}{0}}}{\huxtpad{6pt + 1em}\raggedleft \hspace{6pt} {\fontsize{12pt}{14.4pt}\selectfont (4.635)~~~} \hspace{6pt}\huxbpad{6pt}} \tabularnewline[-0.5pt]


\hhline{>{\huxb{255, 255, 255}{0.4}}->{\huxb{0, 0, 0}{0.4}}->{\huxb{0, 0, 0}{0.4}}->{\huxb{0, 0, 0}{0.4}}-}
\arrayrulecolor{black}

\multicolumn{1}{!{\huxvb{0, 0, 0}{0}}l!{\huxvb{0, 0, 0}{0}}}{\huxtpad{6pt + 1em}\raggedright \hspace{6pt} {\fontsize{12pt}{14.4pt}\selectfont N} \hspace{6pt}\huxbpad{6pt}} &
\multicolumn{1}{r!{\huxvb{0, 0, 0}{0}}}{\huxtpad{6pt + 1em}\raggedleft \hspace{6pt} {\fontsize{12pt}{14.4pt}\selectfont 53940~~~~~~~~} \hspace{6pt}\huxbpad{6pt}} &
\multicolumn{1}{r!{\huxvb{0, 0, 0}{0}}}{\huxtpad{6pt + 1em}\raggedleft \hspace{6pt} {\fontsize{12pt}{14.4pt}\selectfont 53940~~~~~~~~} \hspace{6pt}\huxbpad{6pt}} &
\multicolumn{1}{r!{\huxvb{0, 0, 0}{0}}}{\huxtpad{6pt + 1em}\raggedleft \hspace{6pt} {\fontsize{12pt}{14.4pt}\selectfont 53940~~~~~~~~} \hspace{6pt}\huxbpad{6pt}} \tabularnewline[-0.5pt]


\hhline{}
\arrayrulecolor{black}

\multicolumn{1}{!{\huxvb{0, 0, 0}{0}}l!{\huxvb{0, 0, 0}{0}}}{\huxtpad{6pt + 1em}\raggedright \hspace{6pt} {\fontsize{12pt}{14.4pt}\selectfont R2} \hspace{6pt}\huxbpad{6pt}} &
\multicolumn{1}{r!{\huxvb{0, 0, 0}{0}}}{\huxtpad{6pt + 1em}\raggedleft \hspace{6pt} {\fontsize{12pt}{14.4pt}\selectfont 0.849~~~~} \hspace{6pt}\huxbpad{6pt}} &
\multicolumn{1}{r!{\huxvb{0, 0, 0}{0}}}{\huxtpad{6pt + 1em}\raggedleft \hspace{6pt} {\fontsize{12pt}{14.4pt}\selectfont 0.000~~~~} \hspace{6pt}\huxbpad{6pt}} &
\multicolumn{1}{r!{\huxvb{0, 0, 0}{0}}}{\huxtpad{6pt + 1em}\raggedleft \hspace{6pt} {\fontsize{12pt}{14.4pt}\selectfont 0.851~~~~} \hspace{6pt}\huxbpad{6pt}} \tabularnewline[-0.5pt]


\hhline{>{\huxb{0, 0, 0}{0.8}}->{\huxb{0, 0, 0}{0.8}}->{\huxb{0, 0, 0}{0.8}}->{\huxb{0, 0, 0}{0.8}}-}
\arrayrulecolor{black}

\multicolumn{4}{!{\huxvb{0, 0, 0}{0}}l!{\huxvb{0, 0, 0}{0}}}{\huxtpad{6pt + 1em}\raggedright \hspace{6pt} {\fontsize{12pt}{14.4pt}\selectfont  *** p $<$ 0.001;  ** p $<$ 0.01;  * p $<$ 0.05.} \hspace{6pt}\huxbpad{6pt}} \tabularnewline[-0.5pt]


\hhline{}
\arrayrulecolor{black}
\end{tabular}
\end{threeparttable}\par\end{centerbox}

\end{table}
 
\end{verbatim}

FYI - R also recently introduced the gt package, which is worthwhile
exploring too.

\hypertarget{lists}{%
\section{Lists}\label{lists}}

To add lists, simply using the following notation

\begin{itemize}
\item
  This is really simple

  \begin{itemize}
  \tightlist
  \item
    Just note the spaces here - writing in R you have to sometimes be
    pedantic about spaces\ldots{}
  \end{itemize}
\item
  Note that Rmarkdown notation removes the pain of defining
  \LaTeX environments!
\end{itemize}

\hypertarget{conclusion}{%
\section{Conclusion}\label{conclusion}}

I hope you find this template useful. Remember, stackoverflow is your
friend - use it to find answers to questions. Feel free to write me a
mail if you have any questions regarding the use of this package. To
cite this package, simply type citation(``Texevier'') in Rstudio to get
the citation for \protect\hyperlink{ref-Texevier}{Katzke}
(\protect\hyperlink{ref-Texevier}{2017}) (Note that uncited references
in your bibtex file will not be included in References).

\newpage

\hypertarget{references}{%
\section*{References}\label{references}}
\addcontentsline{toc}{section}{References}

\hypertarget{refs}{}
\begin{CSLReferences}{1}{0}
\leavevmode\hypertarget{ref-Texevier}{}%
Katzke, N.F. 2017. \emph{{Texevier}: {P}ackage to create elsevier
templates for rmarkdown}. Stellenbosch, South Africa: Bureau for
Economic Research.

\end{CSLReferences}

\bibliography{Tex/ref}





\end{document}
